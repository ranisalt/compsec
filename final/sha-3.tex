\documentclass[12pt]{article}
\usepackage{sbc/template}
\usepackage{amsmath,array,booktabs,color,graphicx,listings,url}
\usepackage[brazil]{babel}
\usepackage[utf8]{inputenc}

\sloppy

\title{Função de Hash Criptográfica SHA-3}

\author{Ranieri Althoff}

\address{Universidade Federal de Santa Catarina\\
Departamento de Informática e Estatística\\
Segurança em Computação}

\begin{document}

\maketitle

\section{Introdução}\label{sec:firstpage}

Uma função \textit{hash} é uma função que aceita um bloco de dados de tamanho
variável como entrada e produz um valor de tamanho fixo como saída, chamado de
valor de \textit{hash}. Esta função tem a forma:
\begin{center}
    $h = H(M)$
\end{center}

Onde:
\begin{itemize}
	\item h é o valor hash de tamanho fixo gerado pela função hash.
	\item H é função hash que gerou o valor h.
	\item M é o valor de entrada de tamanho variável.
\end{itemize}

Espera-se que uma função \textit{hash} produza valores $h$ que são
uniformemente distribuídos no contra-domínio e que são aparentemente
aleatórios, ou seja, a mudança de apenas um \textit{bit} em $M$ causará uma
mudança do valor $h$. Por esta característica, as funções \textit{hash} são
muito utilizadas para verificar se um determinado bloco de dados foi
indevidamente alterado.

As funções \textit{hash} apropriadas para o uso em segurança de computadores
são chamadas de ``função \textit{hash} criptográfica''. Este tipo de função
\textit{hash} é implementada por um algoritmo que torna inviável
computacionalmente encontrar:
\begin{itemize}
    \item um valor $M$ dado um determinado valor $h$:
        $M | H(M) = h$
    \item dois valores $M_1$ e $M_2$ que resultem no mesmo valor:
        $(M_1, M_2) | H(M_1) = H(M_2)$
\end{itemize}

Os principais casos de uso de funções \textit{hash} criptográficas são:
\begin{itemize}
    \item Autenticação de Mensagens: é um serviço de segurança onde é possível
        verificar que uma mensagem não foi alterada durante sua transmissão e
        que é proveniente do devido remetente.
    \item Assinatura Digital: é um serviço de segurança que permite a uma
        entidade assinar digitalmente um documento ou mensagem.
    \item Arquivo de Senhas de Uma Via: é uma forma de armazenar senhas usando
        o valor \textit{hash} da senha, permitindo sua posterior verificação
        sem a necessidade de armazenar a senha em claro, cifrá-la ou
        decifrá-la.
    \item Detecção de Perpetração ou Infeção de Sistemas: é um serviço de
        segurança em que é possível determinar se arquivos de um sistema foram
        alterados por terceiros sem a autorização dos usuários do sistema.
\end{itemize}

\subsection{Propriedades}

Como observado na seção anterior, uma função \textit{hash} criptográfica
precisa ter certas propriedades para permitir seu uso em segurança de
computadores. Nas seções a seguir estão destacadas algumas dessas propriedades.

Antes, define-se dois termos usados a seguir:
\begin{itemize}
    \item Pré-Imagem: um valor $M$ do domínio de uma função \textit{hash} dada
        pela fórmula $h = H(M)$ é denominado de “pré-imagem” do valor $h$.
    \item Colisão: para cada valor $h$ de tamanho $n$ \textit{bits} existe
        necessariamente mais de uma pré-imagem correspondente de tamanho $m$
        \textit{bits} se $m > n$, ou seja, existe uma ``colisão''.
\end{itemize}

O número de pré-imagens de $m$ \textit{bits} para cada valor $h$ de $n$
\textit{bits} é calculado pela formula $2^{m/n}$. Se permitimos um tamanho em
\textit{bits} arbitrariamente longo para as pré-imagens, isto aumentará ainda
mais a probabilidade de colisão durante o uso de uma função \textit{hash}.
Entretanto, os riscos de segurança são minimizados se a função de \textit{hash}
criptográfica oferecer as propriedades descritas nas próximas seções.

\subsubsection{Resistente a Pré-Imagem}

Uma função \textit{hash} criptográfica é resistente a pré-imagem quando esta é
uma função de uma via. Ou seja, embora seja computacionalmente fácil gerar um
valor $h$ a partir de uma pré-imagem $M$ usando a função de \textit{hash}, é
computacionalmente inviável gerar uma pré-imagem a partir do valor $h$.

Se uma função \textit{hash} não for resistente à pré-imagem, é possível atacar
uma mensagem autenticada $M_1$ para descobrir o valor secreto $S$ usado na
mensagem, permitindo assim ao perpetrante enviar uma outra mensagem $M_2$ ao
destinatário no lugar do remetente sem que o destinatário perceba a violação da
comunicação. O ataque ocorre da seguinte forma:
\begin{itemize}
    \item O perpetrante tem conhecimento do algoritmo de \textit{hash} usado na
        comunicação entre as partes.
    \item Ao escutar a comunicação, o perpetrante descobre qual é a mensagem
        $M$ e o valor de \textit{hash} $h$.
    \item Visto que a inversão da função de \textit{hash} é computacionalmente
        fácil, o perpetrante calcula $H^{-1}(h)$.
    \item Como $H^{-1}(h) = S || M$, o perpetrante descobre $S$.
\end{itemize}

Desta forma, o perpetrante pode utilizar a chave secreta $S$ no envio de uma
mensagem $M_2$ para o destinatário sem que este perceba a violação.

\subsubsection{Resistente a Segunda Pré-Imagem}

Uma função \textit{hash} criptográfica é resistente a segunda pré-imagem
quando esta função torna inviável computacionalmente encontrar uma pré-imagem
alternativa que gera o mesmo valor \textit{h} da primeira pré-imagem.

Se uma função de \textit{hash} não for resistente a segunda pré-imagem, um
perpetrante conseguirá substituir uma mensagem que utiliza um determinado valor
de \textit{hash}, mesmo que a função de \textit{hash} seja de uma via, ou seja,
resistente a pré-imagem.

\subsubsection{Resistente a Colisão}

Uma função \textit{hash} criptográfica é resistente a colisão quando esta
tornar inviável computacionalmente encontrar duas pré-imagens quaisquer que
possuam o mesmo valor de \textit{hash}. Neste caso, diferentemente da
resistência a segunda pré-imagem, não é dado uma pré-imagem inicial para a qual
precisa se achar uma segunda pré-imagem, mas é suficiente encontrar duas
pré-imagens quaisquer tal que $H(M_1) = H(M_2)$.

Quando uma função \textit{hash} é resistente a colisão, está é consequente
resistente a segunda pré-imagem. Porém, nem sem sempre uma função resistente a
segunda pré-imagem será resistente a colisão. Por isto, diz-se que uma função
\textit{hash} resistente a colisão é uma função de \textit{hash} forte.

Se uma função \textit{hash} não for resistente a colisão, então é possível para
uma parte forjar a assinatura de outra parte. Por exemplo, se Alice deseja que
Bob assine um documento dizendo que deve 100 reais a ela, caso Alice saiba que
um documento contendo o valor de 1000 reais contém o mesmo valor de
\textit{hash} que o documento original, Alice pode fazer com que Bob seja
responsável por uma dívida maior que a original, pois a assinatura valerá para
ambos os documentos.

\subsubsection{Uso das Propriedades de Funções \textit{Hash}}

Abaixo, temos uma tabela que mostra quais propriedades das funções
\textit{hash} são necessárias para alguma das aplicações de segurança de
computadores:

\begin{center}
    \renewcommand{\arraystretch}{1.2}
    \newcolumntype{M}[1]{>{\centering\arraybackslash}m{#1}}
    \begin{tabular}{ M{3.6cm} | M{2.4cm} | M{3.6cm} | M{2.4cm} }
        Aplicação & Resistente a Pré-Imagem & Resistente a Segunda Pré-Imagem & Resistente a Colisão \\ \midrule
        Autenticação de Mensagens & X & X & X \\ \midrule
        Assinatura Digital & X & X & X \\ \midrule
        Infecção de Sistemas & & X & \\ \midrule
        Arquivo de Senhas de Uma Via & X & & \\
    \end{tabular}
\end{center}

No caso da infecção de sistemas, não há problema em usar uma função de
\textit{hash} com fácil inversão, pois não é necessário embutir um valor
secreto na geração do valor de \textit{hash} de um arquivo. Já, num arquivo de
\textit{hash} de senhas, a inversão permitiria descobrir a senha a partir do
valor de \textit{hash}.

Se a função de \textit{hash}, porém, permitir o descobrimento de uma segunda
pré-imagem, seria possível infectar um arquivo de um sistema sem detecção, pois
seu valor de \textit{hash} não mudaria. Isto não seria um problema para um
arquivo de \textit{hash} de senhas, pois o perpetrante não possui a senha, que
é a primeira pré-imagem e, portanto, não teria condições de descobrir a segunda
pré-imagem.

%\bibliographystyle{sbc/sbc}
%\bibliography{sha-3}

\end{document}
