\section{Implementação}

O algoritmo foi implementado em \textbf{C++11} em uma classe \textit{template},
que pode ser configurada para diferentes capacidades e tamanhos de saída em
tempo de compilação. Dessa forma, é possível declarar diferentes instâncias de
\Keccak, como será mostrado após o código. Além disso, se faz necessária uma
função de rotação lógica de inteiros, denominada \texttt{rotl}:

\lstinputlisting[language={[11]C++},firstline=4,lastline=182]{Keccak.h}

As cinco funções de transformação $\theta$, $\rho$, $\pi$, $\chi$ e $\iota$
foram implementadas como métodos que recebem o estado atual e retornam o estado
transformado, com a última ($\iota$) também recebendo o gerador das constantes
de cada rodada.

Além disso, a classe funciona como uma máquina de estados, que está
``absorvendo'' (\texttt{ABSORBING}) ou ``espremendo'' (\texttt{SQUEEZING}),
necessariamente nesta ordem. Um objeto \texttt{Keccak} inicializa com a fase
\texttt{ABSORBING} e muda para a fase \texttt{SQUEEZING} quando a função de
padding é executada.

Além disso, uma função auxiliar \texttt{squeeze\_more} gera mais saída, para
uso no caso das funções de saída extendida, transformando o estado conforme
necessário e armazenando no \textit{stream} \texttt{output}, que a função que
retorna o hash consulta.

O atributo \texttt{npos} possui dois significados: na fase de absorção,
representa quantos \textit{bytes} já foram inseridos no estado; após, informa
quantos \textit{bytes} estão disponíveis em \textit{output}, uma vez que
\textit{streams} não possuem um contador de quantidade de dados disponível.

Uma classe \texttt{SHA3} foi declarada contendo uma única função \texttt{hash},
que automaticamente cria uma instância de \texttt{Keccak}, insere a mensagem
no estado, aplica o padding e retorna o \textit{hash} no tamanho desejado,
usando como padrão a proporção indicada pelo \textbf{FIPS 202}.

Uma classe \texttt{SHAKE} foi declarada contendo as funções \texttt{update},
\texttt{finalize} e \texttt{digest}, que inserem mais valores no estado, trocam
o estado da função e extraem uma quantidade variável de \textit{bytes},
respectivamente com assinaturas baseadas em outras bibliotecas do gênero, como
o \textbf{OpenSSL}.

Dessa forma, é trivial declarar funções com os parâmetros definidos pelo NIST:

\lstinputlisting[language={[11]C++},firstline=184,lastline=219]{Keccak.h}

É importante ressaltar que a todas as funções podem ser aplicadas em uma
\textit{lane} em vez de aplicadas \textit{bit} a \textit{bit}, por isso é mais
eficiente trabalhar com inteiros (neste caso, \texttt{uint64\_t}) do que com
\textit{bitsets}. Está disponível no documento do FIPS 202~\cite{fips:2015} e
na referência do \Keccak~\cite{keccak:2011} o pseudocódigo que gera os mesmos
resultados. O código acima completo está disponível no
GitHub~\cite{althoff:2016}.
