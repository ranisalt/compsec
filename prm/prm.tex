\documentclass[12pt]{article}
\usepackage{sbc/template}
\usepackage{amsmath,amssymb,color,graphicx,listings,url}
\usepackage[brazil]{babel}
\usepackage[utf8]{inputenc}

\definecolor{codegreen}{rgb}{0,0.6,0}
\definecolor{codegray}{rgb}{0.5,0.5,0.5}
\definecolor{codepurple}{rgb}{0.58,0,0.82}
\definecolor{backcolour}{rgb}{0.95,0.95,0.92}

\lstdefinestyle{mystyle}{
    commentstyle=\color{codegreen},
    keywordstyle=\color{blue},
    numberstyle=\tiny\color{codegray},
    stringstyle=\color{codepurple},
    basicstyle=\footnotesize\ttfamily,
    breaklines=true,
    numbers=left
}
\lstset{style=mystyle}

\sloppy

\title{Raízes Primitivas Módulo $n$}

\author{Ranieri S. Althoff\inst{1}}

\address{Universidade Federal de Santa Catarina\\
Departamento de Informática e Estatística\\
Segurança em Computação}

\begin{document}

\maketitle

\section{Introdução}\label{sec:firstpage}

Em aritmética modular, uma raiz primitiva módulo $n$ é um número $g$ tal que
todos os números $m \in \{1, 2, \ldots, n\}$ coprimos de $n$ podem ser
expressado na forma $g^{x} \equiv m \pmod{n}$. Por exemplo, $3$ é uma raiz
primitiva de $7$, porque:

\begin{math}
        3^{1} = 3 \equiv 3 \pmod{7} \\
\indent 3^{2} = 9 \equiv 2 \pmod{7} \\
\indent 3^{3} = 27 \equiv 6 \pmod{7} \\
\indent 3^{4} = 81 \equiv 4 \pmod{7} \\
\indent 3^{5} = 243 \equiv 5 \pmod{7} \\
\indent 3^{6} = 729 \equiv 1 \pmod{7}
\end{math}

O período de $3^{k} \pmod{7}$ é $6$ e os restos nesse período são uma
permutação do conjunto de coprimos de 7, implicando que 3 é uma raiz primitiva.
Neste caso, como 7 é primo, todos os números anteriores são seus coprimos.

\section{Definição}

Se $n$ é um inteiro positivo, os inteiros entre $1$ e $n - 1$ que são coprimos
de $n$ formam um conjunto chamado \textbf{grupo multiplicativo módulo $n$} e é
denotado por $\mathbb{Z}_{n}\phantom{}^{x}$, e esse grupo é cíclico se e
somente se $n \in \{2, 4, p^{k}, 2p^{k}\}$ onde $p^{k}$ é uma exponenciação de
um número primo ímpar $p$.

Um número gerador $g$ desse grupo cíclico é chamado de
\textbf{raiz primitiva módulo n}, ou na teoria dos grupos multiplicativos, um
\textbf{elemento primitivo de $\mathbb{Z}_{n}\phantom{}^{x}$}.

A quantidade de elementos em $\mathbb{Z}_{n}\phantom{}^{x}$ é $\phi(n)$, onde
$\phi$ é a função totiente de Euler. O teorema de Euler diz que
$a^{\phi(n)} \equiv 1 \pmod{n}$ para todo $a$ coprimo de $n$, e o menor
expoente de $a$ que é congruente a $1$ módulo $n$ é chamado de
\textbf{ordem multiplicativa} de $a$ módulo $n$.

Se a ordem multiplicativa de um número $a$ módulo $n$ é $\phi(n)$, ou seja, o
menor número $x$ tal que $a^{x} \equiv 1 \pmod{n}$ é $\phi(n)$, então $a$ é uma
raiz primitiva módulo $n$.

\section{Encontrando raízes primitivas}

O algoritmo de encontrar raízes primitivas módulo $n$ consiste em explorar as
propriedades descritas, pois não há uma fórmula genérica para computar as
raízes de números arbitrários. No entanto, é possível encontrar uma raiz
primitiva de forma mais rápida do que força bruta.

Utilizando a inversa da última propriedade descrita, se $a$ é uma raiz
primitiva módulo $n$, então a ordem multiplicativa de $a$ é $\phi(n)$, e isso
pode ser usado para encontrar raízes multiplicativas.

\subsection{Calculando $\phi(n)$}

O primeiro passo é encontrar $\phi(n)$, para então poder testar possíveis
valores de $a$. Para isso, basta calcular quantos números
$k \in {2, 3, ..., n - 1}$ são coprimos de $n$, utilizando o seguinte código:

\begin{lstlisting}[language=Python]
def phi(n):
    # soma 1 para cada i relativamente primo de n
    return sum(gcd(i, n) == 1 for i in range(1, n))
\end{lstlisting}

O que este código faz, descontando o açúcar sintático da linguagem, é somar 1
para cada número $i$ que seja coprimo de $n$, ou seja, é análogo a seguinte
expressão matemática:

\begin{center}
    $\sum_{i=1}^{n-1}f(i)$, tal que $f(i)=\begin{cases}1, \mbox{gcd(i, n)} = 1 \\ 0, \mbox{gcd(i, n)} \not= 1\end{cases}$
\end{center}

Se, por exemplo, utilizamos 9 como $n$, fazemos o somatório dos seguintes
termos:

\begin{itemize}
    \item $f(1) = 1$, pois gcd(1, 9) $= 1$
    \item $f(2) = 1$, pois gcd(2, 9) $= 1$
    \item $f(3) = 0$, pois gcd(3, 9) $= 3$
    \item $f(4) = 1$, pois gcd(4, 9) $= 1$
    \item $f(5) = 1$, pois gcd(5, 9) $= 1$
    \item $f(6) = 0$, pois gcd(6, 9) $= 3$
    \item $f(7) = 1$, pois gcd(7, 9) $= 1$
    \item $f(8) = 1$, pois gcd(8, 9) $= 1$
\end{itemize}

Portanto, $\phi(9) = 6$.

\subsection{Fatores primos de $\phi(n)$}

Depois de encontrado $\phi(n)$, é necessário determinar os fatores primos de
$\phi(n)$, nomeados $p_{1}, \ldots, p_{n}$. Para encontrar os fatores, basta
percorrer $i = 2, \ldots, n-1$ e encontrar um $i$ que divida $n$. Para
verificar sua primalidade, usa-se fatoração integral ou um teste probabilístico
como o de Fermat ou de Miller-Rabin.

\begin{lstlisting}[language=Python]
# gerador de fatores primos de n
def prime\_factors(n):
    # se n for par, gera 2
    if n&1 == 0:
        yield 2

    # gera p para cada p primo em [3, n/2] que divide n
    for p in range(3, n//2 + 1, 2):
        if n%p == 0 and prime(p):
            yield p

    # gera n se n for primo
    if prime(n):
        yield n
\end{lstlisting}

Este algoritmo utiliza um conceito de Python chamado \textbf{função geradora},
que é uma função que produz mais de um resultado com a \textit{keyword}
\texttt{yield}. Em termos simplificados, a cada \texttt{yield} a função retorna
um valor para quem está a executando.

O uso de geradores ficará mais explícito na continuação deste código, mas a
vantagem é que não é necessário usar tanta memória para guardar todos os
fatores primos de $n$ caso seja um número muito grande quando se retorna um por
um.

O primeiro teste verifica o bit menos significativo de $n$, já que este bit
representa a paridade e todo número par (ou seja, cujo bit menos significativo
é $0$), é divisível por $2$. Como $2$ é o único primo par, isso possibilita
diminuir nosso espaço de busca de fatores pela metade, pois não é necessário
buscar fatores pares (eles não são primos).

Além disso, só é necessário verificar fatores até a metade de $n$, porque a
menor composição possível de um número é $2 \times k$, onde $k$ é um inteiro
qualquer e, portanto, igual a $n / 2$. Em Python, $//$ representa divisão
inteira, ou seja, com arredondamento para baixo.

Utilizando o mesmo $n = 9$, esse algoritmo retornaria apenas $3$, porque é o
único fator primo de $9$. Na primeira execução do laço \texttt{for}, se
verifica que $9 = 0\pmod{3}$, ou seja, $3$ divide $9$ exatamente, e $3$ é
primo, portanto é fator primo de $9$. Nenhum outro valor $p$ entre $3$ e
$9/2=4$ cumpre estas condições.

\subsection{Utilizando o teorema de Fermat}

É necessário verificar o teorema de Fermat $a^{x} \equiv 1 \pmod{n}$ para
$a \in {2, \ldots, n-1}$ e $x \in prime\_factors(n)$. Esse passo é simples,
mas como requer laços aninhados para todos os valores de $a$ e $x$, é um passo
de alta complexidade.

É possível reduzir o espaço de busca considerando que, para que $a$ seja uma
raiz primitiva módulo $n$, $a$ e $n$ devem ser coprimos, ou seja,
gcd(a, n) $= 1$.

Um código que satisfaça as constatações acima segue:

\begin{lstlisting}[language=Python]
def primitive\_roots(n):
    phi\_n = phi(n)

    # testa para todos os valores de a (2 ate n-1)
    for a in range(2, n):
        # para m ser raiz primitiva de n devem ser relativamente primos
        if gcd(a, n) != 1:
            continue

        # verifica a\^(phi(n) / p) mod n para cada fator primo p de phi(n)
        if all(pow(a, phi\_n//p, n) != 1 for p in prime\_factors(phi\_n)):
            yield a
\end{lstlisting}

Novamente fazendo uso de geradores para deixar o código mais leve e limpo.
Sendo bastante auto-explicativo, esse código encontra $\phi(n)$ e, para cada
$a \in [2, n-1]$, executa o teorema de Fermat em todos os $x$ fatores primos de
$\phi(n)$.

O açúcar sintático \texttt{fn(x) for x in generator()}, usado na linha 11,
gera \texttt{x} para cada elemento \texttt{x} no gerador (ou seja, mais um uso
de geradores), e a função \texttt{all(iterable)} testa se todos os valores
passados pra ela são \texttt{True}.

Neste caso, $a$ é raiz primitiva de $n$ se, para todo $x$ fator primo de
$\phi(n)$, $a^{\phi(n)/x} \not\equiv 1 \pmod{n}$, que é o que o \texttt{if}
verifica. Caso seja verdade, $a$ é raiz primitiva de $n$, portanto devemos
gerá-lo com \texttt{yield}.

Para $n = 9$, uma de suas raízes primitivas é $2$, porque
$2^{6/3} \equiv 4 \pmod{n}$. O próximo valor possível seria $3$, mas como
gcd($3, 9$) $= 3$, ele não pode ser raiz primitiva de $9$. O algoritmo segue
fazendo essa verificação até $n-1$ e gerando as raizes primitivas de $n$.

\end{document}
