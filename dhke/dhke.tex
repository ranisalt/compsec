\documentclass[12pt]{article}
\usepackage{sbc/template}
\usepackage{amsmath,amssymb,color,graphicx,listings,url}
\usepackage[brazil]{babel}
\usepackage[utf8]{inputenc}

\definecolor{codegreen}{rgb}{0,0.6,0}
\definecolor{codegray}{rgb}{0.5,0.5,0.5}
\definecolor{codepurple}{rgb}{0.58,0,0.82}
\definecolor{backcolour}{rgb}{0.95,0.95,0.92}

\lstdefinestyle{mystyle}{
    commentstyle=\color{codegreen},
    keywordstyle=\color{blue},
    numberstyle=\tiny\color{codegray},
    stringstyle=\color{codepurple},
    basicstyle=\footnotesize\ttfamily,
    breaklines=true,
    numbers=left,
    showstringspaces=false
}
\lstset{style=mystyle}

\sloppy

\title{Acordo de chaves de Diffie-Hellman}

\author{Ranieri S. Althoff\inst{1}}

\address{Universidade Federal de Santa Catarina\\
Departamento de Informática e Estatística\\
Segurança em Computação}

\begin{document}

\maketitle

\section{Introdução}\label{sec:firstpage}

O acordo de chaves de Diffie-Hellman (D-H) é um método de troca de chaves
criptográficas em meio público (desprotegido) e é um dos primeiros protocolos
de chave pública no meio da criptografia.

Tradicionalmente, a comunicação criptografada entre duas pessoas exigia que
elas primeiro escolhessem uma chave e a transmitissem por um meio seguro, como
um papel transportado por um mensageiro confiável, mas o D-H permite que os
lados da comunicação não necessitem se conhecer para estabelecer uma chave
secreta sob um canal inseguro.

\section{Definição}

O acordo de Diffie-Hellman estabelece uma chave secreta que pode ser usada em
comunicação criptografada para transmissão de dados em uma rede pública de
forma segura.

A implementação original do protocolo utiliza raízes primitivas módulo $p$,
onde $p$ é um número primo e $g$ é uma raiz primitiva de $p$. Essa propriedade
permite que a chave secreta possa representar qualquer valor no intervalo de
$1$ a $p-1$. O protocolo funciona da seguinte forma:

\begin{itemize}
    \item Alice e Beto escolhem um $p$ e $g$ respeitando a definição acima.
    \item Alice escolhe um inteiro secreto $a \leq p$ e envia para Beto
        $A = g^{a} \pmod{p}$.
    \item Beto escolhe um inteiro secreto $b \leq p$ e envia para Alice
        $B = g^{b} \pmod{p}$.
    \item Alice computa o segredo $s$ com $s = B^{a} \pmod{p}$.
    \item Beto computa o mesmo segredo $s$ com $s = A^{b} \pmod{p}$.
\end{itemize}

O algoritmo explora as propriedades da aritmética modular para que, mesmo
quando Alice e Beto não sabem o inteiro secreto do parceiro da comunicação, os
dois possam chegar no mesmo segredo $s$ garantidamente pela relação de
equivalência:

\begin{center}
    $A^{b} \pmod{p} \equiv g^{ab} \pmod{p} \equiv g^{ba} \pmod{p} \equiv
    B^{a} \pmod{p}$
\end{center}

Ou de uma forma mais específica e sem variáveis intermediárias:

\begin{center}
    $\left(g^{a} \pmod{p}\right)^{b} \pmod{p} \equiv
    \left(g^{b} \pmod{p}\right)^{a} \pmod{p}$
\end{center}

Neste caso, desde que um atacante não descubra os segredos $a$ e $b$, ele não
consegue descobrir a chave secreta. A segurança do protocolo está na
inexistência de um algoritmo eficiente para descobrir $a$, $b$ e
$g^{ab} \pmod{p}$ a partir de $g^{a} \pmod{p}$ ou $g^{b} \pmod{p}$.

\subsection{Exemplo}

Utilizando, por exemplo $p = 1987$ e $g = 1484$, que podem ser verificados como
primo e raiz primitiva módulo $p$ com os algoritmos do trabalho anterior. Alice
e Beto escolhem $a = 1201$ e $b = 1690$, dois números naturais quaisquer e
menores que $p$.

\begin{itemize}
    \item Alice calcula $A = 1484^{1201} \pmod{1987} = 1842$ e envia para Beto.
    \item Beto calcula $B = 1484^{1690} \pmod{1987} = 1302$ e envia para Alice.
    \item Alice calcula $s = 1302^{1201} \pmod{1987} = 1135$.
    \item Beto calcula $s = 1842^{1690} \pmod{1987} = 1135$.
\end{itemize}

No final das operações, Alice e Beto podem usar a chave secreta $s = 1135$ para
proteger sua transmissão de informação.

Utilizar inteiros muito grandes é uma tarefa complexa, pelo fato de que números
maiores que 32 bits precisam ser convertidos para inteiros de precisão
arbitrária (\textit{long} em Python), que são lentos para realizar mesmo
operações simples, comparando com inteiros puros.

Utilizando ferramentas de \textit{profiling}, é possível verificar que para um
número de 24 bits são necessárias centenas de milhares de operações de
exponenciação modular (o método \texttt{pow}), e em média é testada a
primalidade de dezenas de milhares números, o que leva alguns segundos.

Esse tempo cresce de forma exponencial e para 25 bits são necessários minutos,
além de que certas sequências de números aleatórios são muito mais lentas. O
tempo pode ser muito reduzido utilizando um cache nas funções \texttt{phi} e
\texttt{prime} que são muito utilizadas com os mesmos argumentos:

\begin{lstlisting}[language=Python]
@functools.lru_cache(maxsize=None)
def phi(n):
    # passo base
    if n < 2:
        return 0

    # se n for primo, sera relativamente primo a todos antes dele
    if prime(n):
        return n - 1

    # igual ao if abaixo, apenas evitando testar para os pares tambem
    if (n & 1) == 0:
        m = n >> 1
        return phi(m) << 1 if not (m & 1) else phi(m)

    # testar para todos os primos menores que n, como p nao eh primo
    # havera algum fator entre 3 e n que divide n
    for p in range(3, n, 2):
        if prime(p):
            if n % p:
                continue

            # phi eh multiplicativo, se p divide n (p * o = n) eh
            # possivel utilizar p (menor que n) para calcular phi de n
            o = n // m
            d, partial = math.gcd(m, o), phi(m) * phi(o)
            return partial if d == 1 else partial * d / phi(d)
\end{lstlisting}

\begin{lstlisting}[language=Python]
@functools.lru_cache(maxsize=None)
def prime(p, tests=20):
    # min evita testar muitas vezes para p pequeno
    for _ in range(min(p, tests)):
        if pow(random.randrange(2, p), p - 1, p) != 1:
            return False
    return True
\end{lstlisting}

O método \texttt{randrange} retorna um inteiro no intervalo $[2, p)$, ou seja,
menor que p e maior que 1, utilizando o algoritmo de \textit{Mersenne twister}.
Não foi utilizado o algoritmo desenvolvido no trabalho 1 (o
\textit{linear congruential generator}) por causa da incapacidade deste de
gerar grandes números e pelo \textit{Mersenne twister} já estar embutido de
forma otimizada na linguagem.

O código que executa o acordo de chaves, já omitindo as funções acima
apresentadas e apresentadas nos trabalhos anteriores, é bastante simples. O
código completo se encontra no arquivo \texttt{dhke.py} anexo ao trabalho.

\begin{lstlisting}[language=Python]
# digitos desejados no numero primo
d = 10

# classe que representa um lado da troca de chaves
class Partner:
    def __init__(self, p, g):
        self.p, self.g = p, g
        self.secret = random.randrange(1, p)

    # executa o primeiro passo: A = g^a mod p para Alice
    def first_step(self):
        return pow(self.g, self.secret, self.p)

    # executa o segundo passo: s = B^a mod p para Alice
    def second_step(self, other):
        self.s = pow(other, self.secret, self.p)

# tenta encontrar um primo p de d digitos e sua raiz primitiva g
p = None
while not prime(p):
    p = random.randrange(10 ** (d - 1), 10 ** d)
g = primitive_root(p)

# imprime na tela o primo e sua raiz primitiva, dados da chave publica
print('p = {}, g = {}'.format(p, g))

# cria cada um dos participantes da troca com p e g acima definidos
alice = Partner(p, g)
bob = Partner(p, g)

# calcula o segredo (segundo passo) de acordo com o primeiro passo
bob.second_step(alice.first_step())
alice.second_step(bob.first_step())

# imprime o valor final do segredo para fins de comparacao, deve sempre
# imprimir o mesmo numero para A e B
print('As = {}, Bs = {}'.format(alice.s, bob.s))
\end{lstlisting}

A dificuldade de gerar inteiros de grande tamanho e testar sua primalidade em
tempo viável impossibilita a implementação desse protocolo em Python. Uma
possibilidade é de implementar em uma linguagem mais rápida como C++ e criar
uma interface com Python para utilizar as funções. Outra possibilidade não tão
eficiente seria de utilizar múltiplas threads, mas o fator de redução do
problema não seria tão grande e haveria o \textit{overhead} de criação das
threads.

\section{Segurança do algoritmo}

\subsection{Ataque de \textit{man in the middle}}

Como o algoritmo não possui qualquer forma de autenticação, é possível que um
atacante monitore a troca de mensagens da negociação de chaves entre Alice e
Beto e simule a troca de mensagens entre eles, mas utilizando seu próprio
segredo para manipular as mensagens e obter seu conteúdo.

Essa falha pode ser contornada utilizando um canal seguro já estabelecido para
troca das mensagens na negociação, sendo então suscetível apenas se o canal
seguro for suscetível ao ataque também.

\subsection{$g$ raiz primitiva módulo $p$}

Como constatado, existe a necessidade de que $g$ seja raiz primitiva de $p$
para que a troca de chaves funcione corretamente. Na situação em que $g$ não é,
o conjunto gerado por $g$ (isto é, os valores de $g^{x} \pmod{p})$ para
$0 < x < p$) não é o maior conjunto possível (de tamanho $p - 1$), o que seria
ideal para o funcionamento do algoritmo.

Para contornar este problema, o número primo $p$ escolhido deve ser
suficientemente grande, de forma que na situação em que $g$ não é sua raiz
primitiva, o conjunto gerado por $g$ ainda seja suficientemente grande para que
não interfira na segurança do algoritmo.

\end{document}
